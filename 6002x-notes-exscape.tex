\author{Thomas Backman, exscape@gmail.com}
\date{\today}
\title{6.002x notes}

% Use a nice font
\documentclass[12pt,a4paper]{report}

% Custom package!
% For strikethrough.
\usepackage{cancel}

% Create clickable URLs
%\usepackage{url}

% Multiline comments
%\usepackage{verbatim}

% Aligned equations, possibly more later
\usepackage{amsmath}

% Links in the table of contents
\usepackage[colorlinks=true, urlcolor=blue,linkcolor=red]{hyperref}

% Fix margins and indentation
\usepackage[cm]{fullpage}
\addtolength{\oddsidemargin}{-0.75cm}
\addtolength{\evensidemargin}{-0.75cm}
\addtolength{\topmargin}{-0.5cm}
\setlength{\parindent}{0in}

% For circuit diagrams
\usepackage{siunitx}
\usepackage[american,siunitx]{circuitikz}

\begin{document}

\maketitle

\tableofcontents

%%%%%%%%%%%%%%%%%%%%%%%%
%% Formulas and such %%%
%%%%%%%%%%%%%%%%%%%%%%%%

\chapter{Formulas and such}

\section{Intro}

This chapter contains several common formulas etc., without any further explanation of how they are derived.

\section{Digital}

\subsection{Noise margins}
\begin{align*}
  NM_H &= V_{OH} - V_{IH}\\
  NM_L &= V_{IL} - V_{OL}\\
\text{Forbidden region} &= V_{IH} - V_{IL}
\end{align*}

\section{MOSFETs}

\subsection{SCS model}
\[ 
 i_{DS} = \begin{cases}
   \frac{K}{2}(v_{GS} - V_T)^2 & \text{for $v_{GS} \ge V_T, v_{DS} \ge v_{GS} - V_T$} \\
   0 & \text{for $v_{GS} < V_T$}
 \end{cases}
\]
The above covers saturation and cutoff regions only (the SCS model).

\subsection{SR model}

\[ 
  i_{DS} = \begin{cases}
  \frac{V_{DS}}{R_{ON}} & \text{for $V_{GS} \ge V_T$} \\
  0 & \text{otherwise}
  \end{cases}
\]

Alternatively:
\[ 
  R_{DS} = \begin{cases}
  R_{ON} & \text{for $V_{GS} \ge V_T$} \\
  \infty & \text{otherwise}
  \end{cases}
\]

Used for digital circuits only (in this course).

\section{State devices / energy storage devices}
This section assumes time-invariant devices, i.e. capacitance/inductance is a fixed value, and not a function of time.

\subsection{Capacitors}

The current through a capacitor is a function of the \emph{rate of change} of voltage:

\[ i(t) = C \frac{dv(t)}{dt} \]

To find the voltage over a capacitor, we need to know its full history:

\[ v(t) = \frac{1}{C} \int_{-\infty}^{t} i(t) dt \]

... or we can simply do it by knowing the current through it between $t_1$ and $t_2$, plus the initial voltage:

\[ v(t_2) = \frac{1}{C} \int_{t_1}^{t_2} i(t) dt + v(t_1) \]

The energy stored in a capacitor is:

\[ E = \frac{1}{2} C v^2 \]

\subsection{Inductors}

The voltage over an inductor is a function of the \emph{rate of change} of current:

\[ v(t) = L \frac{di(t)}{dt} \]

To find the current through a capacitor, we need to know its full history:

\[ i(t) = \frac{1}{L} \int_{-\infty}^t v(t) dt \]

... or we can simply do it by knowing the voltage over it between $t_1$ and $t_2$, plus the initial current through it:

\[ i(t_2) = \frac{1}{L} \int_{t_1}^{t_2} v(t) dt + i(t_1) \]

The energy stored in an inductor is:

\[ E = \frac{1}{2} L i^2 \]

\newpage

\section{Components in series and parallel}

Series resistances add:

\[ R_s = R_1 + R_2 + R_3 + \cdots \]

Parallel resistances diminish:

\[ R_p = \frac{1}{ \frac{1}{R_1} + \frac{1}{R_2} + \frac{1}{R_3} + \cdots} \]

Series capacitances diminish:

\[ C_s = \frac{1}{ \frac{1}{C_1} + \frac{1}{C_2} + \frac{1}{C_3} + \cdots} \]

Parallel capacitances add:

\[ C_p = C_1 + C_2 + C_3 + \cdots \]

Series inductances add:

\[ L_s = L_1 + L_2 + L_3 + \cdots \]

Parallel inductances diminish:

\[ L_p = \frac{1}{ \frac{1}{L_1} + \frac{1}{L_2} + \frac{1}{L_3} + \cdots} \]

For the special case of two components of the diminshing type (x being a dummy variable):

\[ \frac{1}{ \frac{1}{x_1} + \frac{1}{x_2} } = \frac{x_1 \cdot x_2}{x_1 + x_2} \]

%

%%%%%%%%%%%%%%%%%%%%%%%%
%%% Circuit analysis %%%
%%%%%%%%%%%%%%%%%%%%%%%%

\chapter{Circuit analysis}

\section{Thevenin equivalent circuits}

Say we have an capacitor circuit to analyze:

\begin{circuitikz}[scale=1.2]
\draw (0,0) node [ground] {} to [V=$V_S$] (0,3)
                             to [R=$R_1$]  (3,3)
                             to [R=$R_2$]  (6,3)
                             to [C=$C$, v=$v_C$]   (6,0);

\draw (3,3)                  to [R=$R_3$]  (3,0);
\draw (0,0)                  to           (6,0);
\end{circuitikz}
\\

Since this is a linear network, we can simplify it by calculating its \emph{Thevenin equivalent}. Consider the network as seen from the port where the capacitor is attached:

\begin{circuitikz}[scale=1.2]
\draw (0,0) node [ground] {} to [V=$V_S$] (0,3)
                             to [R=$R_1$]  (3,3)
                             to [R=$R_2$, -o]  (6,3);

\draw (3,3)                  to [R=$R_3$]  (3,0);
\draw (0,0)                  to [short, -o]         (6,0);
\draw (6,0)                  to [open, v>=$V_{TH}$] (6,3);
\end{circuitikz}
\\

$V_{TH}$, the open circuit voltage, will be given by the voltage divider formed by $R_3$ and $R_1$:

\[ V_{TH} = \frac{R_3}{R_1 + R_3} \cdot V_S \]

Since no current flows at the port (for the \emph{open circuit} voltage!), $R_2$ doesn't contribute at all.

We also want to measure the resistance ``looking in'' to this port; this will be the Thevenin resistance $R_{TH}$. To do this, we turn off all \emph{independant} voltage and current sources, by replacing all current sources with \emph{opens} and all voltage sources with \emph{short circuits}.\\
Leave dependant sources in the circuit!

\begin{circuitikz}[scale=1.2]
\draw (0,0)                  to [short] (0,3)
                             to [R=$R_1$]  (3,3)
                             to [R=$R_2$, -o]  (6,3);

\draw (3,3)                  to [R=$R_3$]  (3,0);
\draw (0,0)                  to [short, -o]         (6,0);
\draw (6,0)                  to [open] (6,3);
\end{circuitikz}

\[ R_{TH} = R_2 + (R_1 || R_3) = R_2 + \frac{R_1 \cdot R_3}{R_1 + R_3} \]

Now that we know the Thevenin voltage $V_{TH}$ and the Thevenin resistance $R_{TH}$, we can replace the circuit with a voltage source of voltage $V_{TH}$ volts in series with a resistor of value $R_{TH}$ ohm, and place the capacitor back into the circuit:

\begin{circuitikz}[scale=1.2]
\draw (0,0) node [ground] {} to [V=$V_{TH}$] (0,3)
                             to [R=$R_{TH}$] (3,3)
                             to [C=$C$, v=$v_C$]      (3,0);
\draw (0,0)                  to              (3,0);
\end{circuitikz}
                             
Our previous circuit has now turned into a simple RC circuit, which is easier to analyze. See the chapter on RC circuits.\\

As a side note, another way of measuring the Thevenin resistance is to short circuit the output node, calculate/measure the short-circuit current (with all sources left intact, of course), and calculate $R_{TH}$ as $\frac{V_{TH}}{I_{SC}}$.\\

In summary:

$\bullet$ Calculate/measure the open circuit voltage $V_{TH}$ at the port\\
$\bullet$ Turn off all independent sources (make short circuits of voltage sources, and open circuits of current sources), but leave dependant sources intact\\
$\bullet$ Calculate/measure the resistance $R_{TH}$ at the port terminal pair\\
$\bullet$ Replace the original circuit with a series circuit of a voltage source (voltage $V_{TH}$ volts), a resistor (resistance $R_{TH}$ ohm) and the element you want to analyze.

\newpage

\section{Norton equivalent circuits}
Nortan equivalent circuits are very similar to Thevenin equivalents, but use a \emph{current source} in \emph{parallel} with a resistor rather than a \emph{voltage source} in \emph{series} with a resistor.

To convert a circuit to its Norton equivalent:

$\bullet$ Calcurate/measure the \emph{short circuit current}, i.e. the current that would flow through the output port if we were to short-circuit it. The result is the Norton current $I_N$.\\
$\bullet$ Turn off all independent sources (make short circuits of voltage sources, and open circuits of current sources), but leave dependant sources intact.\\
$\bullet$ Calculate/measure the resistance at the port terminal pair; the result is the Norton resistance $R_N$.\\
$\bullet$ Replace the original circuit with a parallel circuit of a current source (current $I_N$ amperes), a resistor (resistance $R_N$ ohm) and the element you want to analyze.\\

\begin{circuitikz}[scale=1.2]
\draw (0,0) node [ground] {} to [I=$I_N$] (0,3);
\draw (0,3) to (3,3);
\draw (3,3) to [R=$R_N$] (3,0);
\draw (6,3) to [C=$C$, v=$v_C$] (6,0);
\draw (6,0) to (0,0);
\draw (6,3) to (3,3);
\end{circuitikz}
\\

Note that since the method for calculating the equivalent resistance is identical for the Thevenin and Norton methods, $R_N = R_{TH}$.
It is easy to convert between one and the other:

\[ R_N = R_{TH} \]
\[ I_N = \frac{V_{TH}}{R_{TH}} \]
\[ V_{TH} = I_N \cdot R_N \]

%%%%%%%%%%%%%%%%%%%%%%%%%%%
%%% Small signal method %%%
%%%%%%%%%%%%%%%%%%%%%%%%%%%

\chapter{Small signal method}
\section{Deriving small signal models}

For a device with $i_D = f(v_D)$, the small signal current $i_d$ is given by

\[ \underbrace { \left. \frac{\partial f(v_D)}{\partial v_D} \right|_{v_D=V_D} }_{\displaystyle g_m} \cdot v_d \]

where $v_d$ is the small signal input voltage.\\
In other words, take the partial derivative of the V-I relation, with respect to the voltage. That gives $g_m$, the transconductance. The transconductance multiplied by the small signal input voltage $v_d$ gives the small signal current.\\

As an example, a MOSFET in the saturation region has $i_{DS} = f(v_{GS}) = \frac{K}{2}(v_{GS} - V_T)^2$:

\[ i_{ds} = \underbrace { \left. \frac{ \partial \frac{K}{2}(v_{GS} - V_T)^2 }{\partial v_{GS}} \right|_{v_{GS}=V_{GS} }}_{\displaystyle g_m} \cdot v_{gs} = \underbrace{ K(V_{GS} - V_T)}_{\displaystyle g_m} \cdot v_{gs} \]

Note the difference between $v_{GS}$ (the total gate-to-source voltage), $V_{GS}$ (the DC bias voltage) and $v_{gs}$ (the small signal / incremental voltage). \\
Also note that the value of $g_m$ depends not only on the MOSFET parameters $K$ and $V_T$, but also on the user-chosen DC bias voltage $V_{GS}$.\\

Here's a table of common circuit elements and their small signal equivalents:\\

\begin{tabular}{| l | l |}
\hline
Device & Small signal replacement \\ \hline
Resistor, R $\Omega$        & Resistor, R $\Omega$ \\ \hline
Voltage source, V volts     & Short circuit \\ \hline
Current source, I amps      & Open circuit \\ \hline
MOSFET in saturation region & VCCS, $i_{ds} = g_m v_{gs}$, $g_m = K(V_{GS} - V_T)$ \\ \hline 
MOSFET with gate/drain tied together & Resistor, $\displaystyle \frac{1}{K(V_{DS} - V_T)} \Omega$ (for $v_{DS} > V_T$) \\ \hline
\end{tabular}

\newpage

\section{Multivariable small signal models}
In some cases, it might be necessary to make a small signal model of a device where the current depends on more than one variable. An example (that will be used here) in 6.002x is the (probably hypothetical) ``NewFET'' in the week 6 homework.\\
In this case, we take the partial derivative of the gate-to-source voltage at the bias point times the small signal voltage $vgs$, \emph{plus} the partial derivative of the drain-to-source voltage at the bias point times the small signal voltage $vds$.

First, the properties of the NewFET:

\[ 
iDS = \begin{cases}
  0 & \text{for $v_{GS} < V_T$} \\
  K(v_{GS} - V_T) {v_{DS}}^2 & \text{for $v_{GS} \ge V_T$}
\end{cases}
\]
\\
The small signal model will look like this:

\begin{circuitikz}[scale=1.2]
\draw (3,2.5) to [cI=$g_m v_{gs}$, *-*] (3,0);
\draw (5,2.5) to [R=$r_o$, *-*] (5,0);
\draw (5,2.5) to [short, i<=$i_{ds}$] (6,2.5) node {\quad D};
\draw (5,0) to (6,0) [short] node {\quad S};
\draw (7,2.5) to [open, v=$v_{ds}$] (7,0);
\draw (3,2.5) to (5,2.5);
\draw (0,0) to (5,0);
\draw (-0.25,0) node {S} [open] to (0,0);
\draw (0,2.5) [open, v=$v_{gs}$, *-*] to (0,0);
\draw (-0.25,2.5) node {G} [open] to (0,2.5);

\end{circuitikz}
\\

The small signal current $i_{ds}$ will depend on both $v_{gs}$ and $v_{ds}$, in this manner:

\[ 
  i_{ds} = v_{gs} \underbrace{ \left. \frac{\partial i_D}{\partial v_{GS}} \right|_{v_{GS}=V_{GS}}}_{\displaystyle g_m} + v_{ds} \underbrace{ \left. \frac{\partial i_D}{\partial v_{DS}} \right|_{v_{DS}=V_{DS}}}_{\displaystyle 1/r_o}
\]

$g_m$ will be the regular transconductance, calculated the same as with MOSFETs (see the above section):

\[ 
  g_m = \left. \frac{\partial i_D}{\partial v_{GS}} \right|_{v_{GS}=V_{GS}} = \left. \frac{\partial K(v_{GS} - V_T) {v_{DS}}^2}{ \partial v_{GS}} \right|_{v_{GS}=V_{GS}} = K {V_{DS}}^2
\]

$r_o$ will be the \emph{reciprocal} of the partial of $i_D$ with respect to $v_{DS}$:\\
(In other words, the partial will give a conductance, and we want a resistance.)

\[ 
  \frac{1}{r_o} = \left. \frac{\partial i_D}{\partial v_{DS}} \right|_{v_{DS}=V_{DS}} = \left. \frac{\partial K(v_{GS} - V_T) {v_{DS}}^2}{ \partial v_{DS}} \right|_{v_{DS}=V_{DS}} = 2 K V_{DS} (V_{GS} - V_T)
\]
So
\[ r_o = \frac{1}{2 K V_{DS} (V_{GS} - V_T)} \]

From these equations and the circuit diagram, we see that 
\[
 i_{ds} = \frac{v_{ds}}{r_o} + g_m v_{gs} = v_{ds} \cdot 2 K V_{DS} (V_{GS} - V_T) + K {V_{DS}}^2 \cdot v_{gs} 
 \]

Note that, although the expression contains a square term (${V_{DS}}^2$), it is still linear, as the square term is a constant - the bias voltage $V_{DS}$ should not change, or the entire small signal model will be incorrect either way.

So, the result is not quite the simplest of expressions, but when the bias constants are replaced with their actual bias values, the result is of the form

\[ i_{ds} = C_{1} v_{ds} + C_{2} v_{gs} \]

where $C_1$ and $C_2$ are constants.

%%%%%%%%%%%%%%%%%%%%%%%%%%%%
%%% First-order circuits %%%
%%%%%%%%%%%%%%%%%%%%%%%%%%%%

\chapter{First-order circuits}
\section{Series RC circuits}

\begin{circuitikz}[scale=1.2]
\draw (0,0) node [ground] {} to [V=$V_S$] (0,3)
					  to [R=$R$]     (3,3)
					  to [C=$C$, v<=$v_C$]	(3,0);
\draw (3,0) to (0,0);
\end{circuitikz}
\\

We start off by writing down a KCL equation for the unknown node voltage $v_C$:\\
\[ \frac{v_C - V_S}{R} + C \frac{dv_C}{dt} = 0 \]

Rewrite the equation to get it in the form we prefer:

\[ RC \frac{dv_C}{dt} + v_C = V_S \]

To solve this first-order linear differential equation, we'll use the method of particular and homogeneous solutions, where we need to find two solutions to the differential equation: the first (the \emph{particular} solution) is \emph{any} solution that makes the equation true:

\[ RC \frac{dv_{Cp}}{dt} + v_{Cp} = V_S \]

We see here that if we pick $v_{Cp} = V_S$, where $V_S$ is a constant, thus making $\frac{dV_S}{dt} = 0$, this equation is indeed true; since the first term becomes 0, all that remains is $V_S = V_S$ - which is clearly true!\\
Thus, we've found the particular solution:

\[ v_{Cp} = V_S \]

The next step in this method is to find a solution to the homogeneous equation, where $V_S$ (the ``input drive'') is zero:

\[ RC \frac{dv_{Ch}}{dt} + v_{Ch} = 0 \]

We need a function such that its derivative is the function itself times a constant. $e^x$ comes to mind - more specifically, the solution will have some (still unknown) coefficients $A$ and $s$, such that:

\[ v_{Ch} = Ae^{st} \]

We substitute $Ae^{st}$ into the homogeneous equation and end up with:

\[ RC \frac{dAe^{st}}{dt} + Ae^{st} = 0 \]

We calculate the derivative and replace the $\frac{d}{dt}$ term with it:

\[ RCAs e^{st}  +Ae^{st} = 0 \]

Divide both sides by $Ae^{st}$:

\[ RCs + 1 = 0 \]
\[ RCs = -1 \]
\[ s = -\frac{1}{RC} \]

We've thus found one of our two constants.\\
The total solution to the differential equation will be the \emph{sum} of the particular and homogeneous solutions, so the next step is to add them together:

\[ v_C(t) = v_{Cp}(t) + v_{Ch}(t) \]
\[ v_C(t) = V_S + Ae^{-\frac{1}{RC} \cdot t} \]

All that remains to do is to find the value of the constant $A$. To do so, we substitute $v_C$ for the given initual condition $v_C(0) = V_0$, while setting $t = 0$:

\[ V_0 = V_S + Ae^0 \]
\[ V_0 = V_S + A \]
\[ A = V_0 - V_S \]

We've thus found the full solution:

\[ v_C(t) = V_S + (V_0 - V_S)e^{-\frac{t}{RC}} \]

\newpage

\section{Parallel RC circuit with a current source}

\begin{circuitikz}[scale=1.2]
\draw (0,0) node [ground] {} to [I=$I$] (0,3);
\draw (0,3) to (2,3);
\draw (2,3) to [R=$R$] (2,0);
\draw (2,3) to (4,3);
\draw (4,3) to [C=$C$, v=$v_C$] (4,0);
\draw (4,0) to (0,0);
\end{circuitikz}
\\

To save time (and space): the differential equation we end up with is exactly the same as for the series circuit above, with the sole difference that $IR$ replaces $V_S$, where $I$ is the current source drive current, and $R$ is the parallel resistor's resistance.\\
Since the resulting equation

\[ RC \frac{dv_C}{dt} + v_C = IR \]
is of the same form as the one for the series circuit, the solution is also the same:

\[ v_C(t) = IR + (V_0 - IR)e^{-\frac{t}{RC} } \]

\chapter{Second-order circuits}
\section{Series RLC circuits}

\begin{circuitikz}[scale=1.2]
\draw (0,0) node [ground] {} to [V=$V_S$] (0,3)
                     to [L=$L$, v<=$v_L$]     (2,3)
					  to [R=$R$, v<=$v_R$, i>=$i$]     (4,3)
					  to [C=$C$, v<=$v$]	(4,0);
\draw (4,0) to (0,0);
\end{circuitikz}

We know that the current $i$ relates to the capacitor voltage:

\[ i = C \frac{dv}{dt} \]

Since this is a series circuit, that current goes through all elements.

By KVL, we can add up the voltage drops around the loop, with the proper sign. I'll go clockwise, and start in the bottom left:

\[ -V_S + v_L + v_R + v = 0 \]

If we solve for $V_S$:

\[ V_S = v_L + v_R + v \]

Now, let's use the element laws for the inductor and resistor, and substitute them into the above:

\[ V_S = L \frac{di(t)}{dt} + R i(t) + v(t) \]
\newpage
We know that $i(t) = C \frac{dv}{dt}$, so let's substitute that back in. Let's also differentiate, in case of the inductor, to reduce the mess of nested differentiation operators:

\[ V_S = LC \frac{d^2v(t)}{dt^2} + RC \frac{dv}{dt} + v(t) \]

There we go; we now have a second-order, linear, constant coefficient ordinary differential equation.\footnote{I just love these overly verbose classifications.}\\

We'll use the same method to solve it as we did for the first-order ones, namely the method of particular and homogeneous solutions. First we first the particular solution (\emph{any} solution that makes the equation true), then the homogeneous solution (a solution to the equation with the input drive $V_S$ set to 0), and finally we add the two together to find the total solution.\\

As for the particular solution, just as in the first-order case, we try $v(t) = V_S$ and see that the two derivatives both go to zero (as $V_S$ is a constant), and so we end up with

\[ V_S = V_S \]

which tells us that $V_S$ indeed is a particular solution. On to the homogeneous one, i.e. the solution to 

\[ LC \frac{d^2v(t)}{dt^2} + RC \frac{dv}{dt} + v(t) = 0\]

Again, like in the first-order case, we try the form 

\[ v(t) = Ae^{st} \]

where $A$ and $s$ are constants we'll have to find later. So, we substitute $Ae^{st}$ for $v(t)$, and differentiate where necessary, which gives us

\[ LC As^2 e^{st} + RC As e^{st} + A e^{st} = 0 \]

We can cancel out a ton of the above, by dividing both sides by $A e^{st}$:

\[ LC \cancel{A} s^2 \cancel{e^{st}} + RC \cancel{A}s \cancel{e^{st}} + \cancel{A e^{st}} = 0 \]
What remains is

\[ LC s^2 + RC s + 1 = 0 \]

Divide by LC throughout:

\[ s^2 + \frac{R}{L} s + \frac{1}{LC} = 0 \]

We've thus arrived at the \emph{characteristic equation}, which describes most details of the circuit's behavior. There's a canonical form to write such an equation:

\[ s^2 + 2\alpha s + {\omega_0}^2 = 0 \]

So, in the case of the series RLC circuit, the values of $\alpha$ and $\omega_0$ would have to be

\[ \alpha = \frac{R}{2L} \]
and
\[ \omega_0 = \frac{1}{\sqrt{LC}} \]

$\alpha$ relates to the damping factor of the circuit (sometimes the damping factor, $\displaystyle \zeta = \frac{\alpha}{\omega_0}$, is used) - that is, the ringing second-order circuits can produce will decay faster for a larger value of $\alpha$. Meanwhile, $\omega_0$ is the \emph{undamped resonance frequency} of the circuit (in radians/second), the same as in undriven LC circuits. However, this might not be the frequency of interest in a RLC circuit - more on that later.\\

Let's get back to solving the differential equation. Remember, we only found the homogeneous solution - we still haven't figured out the values of $s$ or $A$.

The next step would be to find the roots of the characteristic equation:

\[ s^2 + 2\alpha s + {\omega_0}^2 + 0 \]

The quadratic formula will work nicely on this. The two roots are:

\[ s_1 = -\alpha + \sqrt{\alpha^2 - {\omega_0}^2} \]
\[ s_2 = -\alpha - \sqrt{\alpha^2 - {\omega_0}^2} \]

The full solution to the homogeneous solution will then be

\[ v_H = A_1 e^{s_1 t} + A_2 e^{s_2 t} \]

However, we've now found $s_1$ and $s_2$, so we can fill them in:
\large
\[ v_H = A_1 e^{(-\alpha + \sqrt{\alpha^2 - {\omega_0}^2}) t} + A_2 e^{(-\alpha - \sqrt{\alpha^2 - {\omega_0}^2}) t} \]
\normalsize

Then, the total solution (prior to finding $A_1$ and $A_2$, and also prior to simplifying) is:

\[ v(t) = V_S + A_1 e^{(-\alpha + \sqrt{\alpha^2 - {\omega_0}^2}) t} + A_2 e^{(-\alpha - \sqrt{\alpha^2 - {\omega_0}^2}) t} \]

After some magic\footnote{Sorry, but I don't quite follow the entire process myself yet. Since this part was cut out from the lectures in order to save time (after ~14 videos going through this process so far), I will do the same.} and making the definition $\omega_d = \sqrt{{\omega_0}^2 - \alpha^2}$, the total solution ends up looking like this, for the case $\omega_0 > \alpha$, i.e. the underdamped case:

\[ v(t) = V_I + K_1 e^{-\alpha t} \cos{\omega_d t} + K_2 e^{-\alpha t} \sin{\omega_d t} \]

Evaluating the above at t = 0 with the initial condition $v(0) = 0$ gives us:

\[ 0 = V_I + K_1 \]
\[ K_1 = -VI \]

Evaluating at t = 0 with the other initial condition, $\displaystyle i(0) = C \frac{dv}{dt} = 0$, gives, after the differentiation and substitution for $t = 0$:

\[ 0 = -K_1 \alpha + K_2 \omega_d \]

We know that $K_1 = -V_I$:

\[ 0 = V_I \alpha + K_2 \omega_d \]
\[ K_2 \omega_d = -V_I \alpha \]
\[ K_2 = -\frac{V_I \alpha}{\omega_d} \]

Thus, finally, the full equation that governs the capacitor voltage of the series RLC circuit, in the underdamped case ($\omega_0 > \alpha$) is:

\[ v(t) = V_I - V_I e^{-\alpha t} \cos{\omega_d t} - \frac{V_I \alpha}{\omega_d} e^{-\alpha t} \sin{\omega_d t} \]

Using the scaled sum of sines trig identity, we can turn this cosine minus sine business into a a single cosine, and end up with:

\[ v(t) = V_I - V_I \frac{\omega_0}{\omega_d} e^{-\alpha t} \cos{(\omega_d t - \arctan{(\frac{\alpha}{\omega_d})})} \]

While the equation is rather involved, it's still very clear that the two main features are a cosine multiplied by a decaying exponential, which will give us exactly the kind of waveform we see in a damped second-order system.

\subsection{Summary}
Since this section was by \emph{far} the longest of this document, I thought a summary of the relevant formulas could be useful. Remember that most of these only apply to series RLC circuits, though some definitions (such as $\omega_d$) are universal.\\

Where v(t) denotes the capacitor node voltage in the circuit shown (far) above - also, remember that this is for the \emph{underdamped} case, i.e. $\displaystyle \omega_0 > \alpha$:

\[ v(t) = V_I - V_I \frac{\omega_0}{\omega_d} e^{-\alpha t} \cos{(\omega_d t - \arctan{(\frac{\alpha}{\omega_d})})} \]

where

\[ \alpha = \frac{R}{2L} \text{ rad/s}\]
\[ \omega_0 = \frac{1}{\sqrt{LC}} \text{ rad/s} \]
\[ \omega_d = \sqrt{{\omega_0}^2 - \alpha^2} \text{ rad/s} \]

Other formulas that may be useful are:
\[ Q = \frac{\omega_0}{2\alpha} \text{ (dimensionless)} \]
\[ \text{Period} = \frac{2\pi}{\omega_d} \text{ seconds} \]
\[ \text{Frequency} = \frac{\omega_d}{2\pi} = \frac{1}{2\pi \sqrt{LC}} \text{ Hz} \]

The quality factor, $Q$, can (for now) be thought of as the approximate number of oscillations that will occur before the ringing dies out due to the damping, though the actual meaning is more precisely defined.

\chapter{Miscellaneous}
\section{Impulses and steps}

\subsection{A note}
For a linear circuit - one with resistors, capacitors and inductors \emph{ONLY} - no sources allowed - an interesting relation between input and output exists. If y(t) is the output for the input x(t), the output for the derivative of x(t) will be the derivative of y(t). The same goes for integration. That is,

\[ x(t) \to y(t) \]
\[ x'(t) \to y'(t) \]
\[ \int x(t) dt \to \int y(t) dt \]

The usefulness for this will soon be clear. In a short preview, we can calculate the behaviour of such a circuit to for example an impulse, and then integrate the response we got; the integrated answer will then be the circuit response to a \emph{step}, as a step is the result of integrating an impulse.

\subsection{Impulses}
Impulses are (theoretical) bursts of an infinite voltage/current over an infinitesimally small time. The Dirac Delta function, $\delta(t)$, is used to describe them. The delta function is defined as\footnote{Perhaps not rigorously, but good enough for our purposes.}

\[ \delta(x) = \begin{cases}
   +\infty & x = 0 \\
   0       & x \neq 0
   \end{cases}
\]
In other words, it is zero everywhere except at $x = 0$, where it's infinite. There is an additional identity it is constrained to, however:

\[ \int_{-\infty}^{\infty} \delta(x) dx = 1 \]
In other words, the area under the curve is exactly 1. Mathematically, the delta function is not a proper function, but can be defined as a distribution.

In a parallel, current source-driven RC circuit, a current impulse at $t = 0$ delivers 100\% of its charge to the capacitor, and essentially creates an initial condition for $t = 0^{+}$. Assuming the capacitor has no state, the capacitor voltage after the impulse will be 

\[ v_C(0^{+}) = \frac{Q}{C} \]

where Q is the area of the impulse, i.e. the current source output is given by

\[ Q \delta(t) \], so Q is in coulombs, the SI unit for electric charge. The unit is such that

\[ \frac{1\text{ coulomb}}{1\text{ farad}} = 1\text{ volt} \]

In such a circuit, the capacitor voltage (and thus the entire circuit's voltage, as there is only 1 node besides ground) for $t > 0^{+}$ will be given by

\[ v_C(t) = \frac{Q}{C} e^{-\frac{t}{RC}} \]

The dual of this circuit is the series RL circuit, with a voltage source, a resistance and an inductance. In this case, we have a voltage impulse, that delivers a flux linkage $\Lambda$ to the inductor. As above, this will essentially create an initial condition, this time for the inductor current. Assuming the inductor has no state, the current through it will be:

\[ i_L(0^{+}) = \frac{\Lambda}{L} \]

where $\Lambda$ is the area of the impulse, i.e. the voltage source output is given by

\[ \Lambda \delta(t) \], so $\Lambda$ is in webers, the SI unit for magnetic flux. The unit is such that
\[ \frac{1\text{ weber}}{1\text{ henry}} = 1\text{ ampere} \]

Again, an in the previous case (the circuits are duals, after all), the inductor current for $t > 0^{+}$ is given by

\[ i_L(t) = \frac{\Lambda}{L} e^{-\frac{R t}{L}} \] 

\subsection{Steps}
Steps are the result of integrating an impulse. Steps are likely more intuitive than impulses. The unit step function (or the \emph{Heaviside step function}) has the definition

\[ u(t) = 
  \begin{cases}
   0  & t < 0 \\
   1  & t \ge 0
  \end{cases}
\]
In words, it ``turns on'' at $t = 0$, and so is a useful model for things such as a voltage source turning on at $t = 0$. Such a source could be written as

\[ V_I u(t) \]
where $V_I$ is the voltage of the source when turned on.\\

As noted in the introduction section, a step is the result of integrating an impulse. Thus, we can calculate the circuit response to a step by calculating the response to an impulse, and then integrate the answer.\\
In a similar manner, we can calculate the response to a step, and differentiate the answer to find the response to an impulse.

\subsection{Shifts}
The above functions can be shifted to start at times other than $t = 0$ by shifting them.\\
For example, while the unit step function $u(t)$ ``turns on'' at $t = 0$, we can make it turn on at time $t = 5$ by shifting it, like so: $u(t - 5)$\\
The above function will have $t - 5$ be negative until $t = 5$ where it becomes positive, and the unit step function changes its output from 0 to 1.\\
The exact same principle applies to the unit impulse function $\delta(t)$.

In general, for the change to happen at time $t = T$, use $u(t - T)$ or $\delta(t - T)$.

\end{document}